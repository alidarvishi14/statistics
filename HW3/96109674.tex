\documentclass[12pt]{article}
\usepackage{HomeWorkTemplate}
\usepackage{circuitikz}
\usepackage{tikz}
\newcommand*\coin[1]{\tikz[baseline=(char.base)]{
            \node[shape=circle,draw,inner sep=2pt,minimum size=10mm] (char) {#1};}}
\usepackage{float}
\usepackage{xepersian}
\usetikzlibrary{arrows,automata}
\usetikzlibrary{circuits.logic.US}
\settextfont{B Nazanin}
\newcounter{problemcounter}
\newcounter{subproblemcounter}
\linespread{1.2}
\setcounter{problemcounter}{1}
\setcounter{subproblemcounter}{1}
\newcommand{\grade}[1]{\textbf{(#1 نمره)}}
\newcommand{\problem}[1]
{
\subsection*{
مسأله‌ی
\arabic{problemcounter} 
\stepcounter{problemcounter}
\setcounter{subproblemcounter}{1}
#1
}
}
\newcommand{\subproblem}{
\textbf{\harfi{subproblemcounter})}\stepcounter{subproblemcounter}
}
\newcommand{\n}{

\null

}
\begin{document}

\handout
{آمار و کاربردها}
{}
{۲۸ آذر 1398}
{سری ۳}
{۵ دی}
علیرضا درویشی................................................................................................................................................................................. ۹۶۱۰۹۶۷۴
\subsection*{مسأله ۱}
با توجه به اینکه این بازه ی اطمینان با محاسباتی روی تعدادی متغیر تصادفی بدست آمده است، می توانیم متغیر تصادفی $z$ را تعریف کنیم که اگر  $\theta$ در بازه ی اطمینان قرار داشت، $z=1$ و در غیر این صورت $z=0$. حال $Pr(z=1)=0.95$ اگر بازه ی اطمینان ساخته شده، بازه ی اطمینان $95\%$ باشد.

نحوه ی دیگر نگاه کردن به این مسئله این است که اگر تعداد زیادی بازه ی اطمینان $95\%$ بسازیم، فقط $95\%$ از بازه های اطمینان، $\theta $ را دربر می گیرند.
\subsection*{مسأله‌ ۲}
\begin{flushleft}
$
\begin{tabular}{  | c | c | c | c |}
\hline
 $diffrence$ & $spending\ after\ rise$     &        $spending\ before\ rise$   &  \\ \hline
$42$ &  $542$ & $500$ &$ 1 $\\ \hline
 $32$ & $403 $& $371$ & $2$ \\ \hline
$-13$ &$ 591 $& $604$ & $3$ \\ \hline
$-31$&  $39$ & $70$   & $4$ \\ \hline
$11$ & $54$ & $43$   & $5$ \\ \hline
$8.2$ & $-$ & $-$   & $mean$ \\ \hline
$30.42$ & $-$ & $-$   & $s$ \\ \hline
\end{tabular}
$

$
\begin{cases}
H_0& \mu=0 \\
\neg H_0 &\mu\neq 0
\end{cases}
$

$t_{4,0.05}=2.776\Rightarrow(-t_{4,0.05}s/\sqrt{5},t_{4,0.05}s/\sqrt{5})=(-37.7,37.7)$

$
p_{value}=1-Pr(|x|<8.2|\mu=0,\sigma=30.42/\sqrt{5})$

$=1-Pr(|T|<0.60|T \sim t-student(df=4,\mu=0,\sigma=1))=0.58>0.05$

\end{flushleft}
پس فرض صفر رد نمی شود
\subsection*{مسأله‌ی ۳}
الف)
\begin{flushleft}
$
\begin{cases}
H_0& \mu=75 \\
\neg H_0 &\mu= 80
\end{cases}
$

$s=3.20\ , df=9$

$mean=76.91$

$p_{value}=Pr(x>76.91|\mu=75)=Pr(\frac{x-75}{\frac{3.2}{\sqrt{10}}}>0.378125)$

$=Pr(T>1.89|T\sim t-student(df=9,\mu=0,\sigma=1))=0.046<0.05$

\end{flushleft}
پس فرض صفر را می توان رد کرد

ب)
\begin{flushleft}
$T=\frac{76.91-80}{3.20/\sqrt{10}}=-3.05$

$\beta=Pr(T<-3.05|T\sim t-student(df=9))=0.0069$
\end{flushleft}
\subsection*{مسأله‌ی 4}
\begin{flushleft}
$
\begin{cases}
H_0 & \pi_1-\pi_2=0 \\
\neg H_0 & \pi_1-\pi_2 \neq 0
\end{cases}
$

$P_1=\frac{17}{17+584}=0.028,\ P_2=\frac{23}{23+376}=0.058$

$\pi=P_1-P_2\Rightarrow Var(\pi)=Var(P_1)+Var(P_2)\Rightarrow Var(\pi)=\frac{P_1(1-P_1)}{n_1}+\frac{P_2(1-P_2)}{n_2}$

$\Rightarrow \pi \in (-z_{0.05}\sqrt{Var(\pi)},z_{0.05}\sqrt{Var(\pi)}) \rightarrow H_0\ is \ not\ rejected\ with \ 90\% \ confidence $

$z_{0.05}=1.64\Rightarrow 90\% \ confidence \ interval\ :\ (-0.022,0.022),\ \pi=-0.03$

$\Rightarrow H_0\ is\ rejected$
\end{flushleft}
\subsection*{مسأله‌ی ۵}
\begin{flushleft}
$\Pi=0.34$

$
\begin{cases}
H_0 & \pi=\Pi \\
\neg H_0 & \pi < \Pi
\end{cases}
$

$P=0.23$

$Z=\frac{P-\Pi}{\sqrt{\frac{P(1-P)}{n}}}$

$Z=-6.13\Rightarrow p_{value}=Pr(z<-6.13)=4\times 10^{-10}<0.04\Rightarrow H_0\ is\ rejected\ with\ 96\% \ confidence $
\end{flushleft}
پس تحصیل در خارج از کشور (با قطعیت ۹۶ درصد) در احتمال قبولی تاثیر دارد.
\end{document}
