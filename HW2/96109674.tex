\documentclass[12pt]{article}
\usepackage{HomeWorkTemplate}
\usepackage{circuitikz}
\usepackage{tikz}
\newcommand*\coin[1]{\tikz[baseline=(char.base)]{
            \node[shape=circle,draw,inner sep=2pt,minimum size=10mm] (char) {#1};}}
\usepackage{float}
\usepackage{xepersian}
\usetikzlibrary{arrows,automata}
\usetikzlibrary{circuits.logic.US}
\settextfont{B Nazanin}
\newcounter{problemcounter}
\newcounter{subproblemcounter}
\linespread{1.2}
\setcounter{problemcounter}{1}
\setcounter{subproblemcounter}{1}
\newcommand{\grade}[1]{\textbf{(#1 نمره)}}
\newcommand{\problem}[1]
{
\subsection*{
مسأله‌ی
\arabic{problemcounter} 
\stepcounter{problemcounter}
\setcounter{subproblemcounter}{1}
#1
}
}
\newcommand{\subproblem}{
\textbf{\harfi{subproblemcounter})}\stepcounter{subproblemcounter}
}
\newcommand{\n}{

\null

}
\begin{document}

\handout
{آمار و کاربردها}
{}
{15 آبان 1398}
{سری 2}
{17 آبان}
علیرضا درویشی................................................................................................................................................................................. ۹۶۱۰۹۶۷۴
\subsection*{مسأله ۱}
\begin{flushleft}
$Var(X)=E[X^2]-E^2[X]=\int_{-2+\theta}^{2+\theta} \frac{1}{4}x^2dx-\int_{-2+\theta}^{2+\theta} \frac{1}{4}xdx=
\frac{1}{12}((\theta+2)^3-(\theta-2)^3)-\frac{1}{8}((\theta+2)^2-(\theta-2)^2)=
\frac{4}{3}$

$z=\sum_{i=1}^n x_i/n\Rightarrow Var(z)=\frac{1}{n}Var(x)=\frac{4}{3n}$

$\Rightarrow \theta=30\pm1.96 \sqrt{\frac{4}{300}}=30\pm 0.23\  ,\ with\ 95\% \ confidence$
\end{flushleft}

\subsection*{مسأله‌ ۲}
\begin{flushleft}
$Z=\sum_{i=1}^nx_i/n$

$E[Z-\theta]=E[Z]-\theta=\frac{1}{n}E[\sum_ix_i]-\theta=\frac{1}{n}n\theta-\theta=0$

$Var(X)=\theta^2\Rightarrow \sigma_x=\theta$

$SE(Z)=\frac{\sigma_x}{\sqrt{n}}=\frac{\theta}{\sqrt{n}}=\frac{1}{n\sqrt{n}}\sum_{i=1}^nx_i$
\end{flushleft}
\subsection*{مسأله‌ی ۳}
الف)
راه اول:
\begin{flushleft}
$\int_{z=-x}^x \frac{1}{\sqrt{2\pi}}e^{\frac{-z^2}{2}}dz=0.9\Rightarrow x=1.64$

$Var(X)=\theta^2\Rightarrow \sigma_x=\theta \Rightarrow Var(z=\sum_i x_i/n)=\frac{1}{n}\theta^2\Rightarrow \sigma_z=\frac{1}{\sqrt{n}}\theta$

$\theta=\frac{\sum_{i=1}^n x_i}{n}\pm 1.64 \frac{\sum_{i=1}^n x_i}{n\sqrt{n}}\  ,\ with\ 90\% \ confidence$
\end{flushleft}
راه دوم:
\begin{flushleft}
\lr{First let us prove that if X follows an exponential distribution with parameter 2
$\lambda$, then Y = 2$\lambda$X follows an exponential distribution with parameter 1/2, i.e. $\chi^2$. The density function for X is f(x|$\lambda$) = $\lambda e^{-\lambda x }$ if x > 0 and 0 otherwise. It is easy to see the density function for Y is $g(y) = \frac{1}{2} e^{-y/2} $for y > 0, and g(y) = 0 otherwise. Therefore Y has an exponential distribution with parameter 1/2, i.e. chi-square distribution with degree of freedom 2.}

\lr{Now, let us first find a pivot, define}

$$h(X_1,X_2,..,X_n,\lambda)=2\lambda \sum X_i=\sum Y_i$$

\lr{and each $Y_i = 2\lambda X_i$ follows $\chi^2$ distribution, and they are independent. Therefore h follows
$\chi_n^2$ distribution.
Let $\chi^2_n(\alpha/2)$ and $\chi^2_n(1-\alpha/2)$ be the $(\alpha/2)$ × 100-th and $(1-\alpha/2)$ × 100-th percentiles,
respectively. Then}

$$P(\chi_{2n}^2(\alpha/2)\leq2\lambda \sum X_i \leq \chi_{2n}^2(1-\alpha/2))=1-\alpha$$

$$P(\frac{2\sum X_i}{\chi_{2n}^2(1-\alpha/2)} \leq \theta  \leq \frac{2\sum X_i}{\chi_{2n}^2(\alpha/2)})=1-\alpha$$
\end{flushleft}

ب)
\begin{flushleft}
$\theta \in (l(x),\infty), \ with\ 95\% \ confidence$

$\alpha=0.95$

$\alpha=pr(l(x)<\theta)=pr(x<l^{-1}(\theta))=\frac{l^{-1}(\theta)}{\theta}\Rightarrow l^{-1}(\theta)=\alpha\theta\Rightarrow l(x)=\frac{x}{\alpha}=\frac{1}{0.95}x$

$\Rightarrow \theta\geq 1.052x , \ with \ 95\% \ confidence$
\end{flushleft}
\end{document}
